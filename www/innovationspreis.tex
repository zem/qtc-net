\documentstyle{article}

\begin{document}

\begin{abstract}
What if HAM A wants to send a Message to HAM B but he does not have any 
Phone Number, Address, information where his friend is and what activities 
he is involved in? He will send a QTC-NET Telegram in the future. 
\end{abstract}

\section{Goals}

The Goal of this project is to create something that:

\begin{itemize}
	\item should just use the callsigns for addressing, no routing info 
	\item should know when somebody owns more than one callsign
	\item is widely accepted, relieable
	\item should not fail in case of a catastrophic event
	\item can legally distributed over ham radio links
	\item provides more data-safety than a late 80th solution
\end{itemize}

This is a huge task. To be widely accepted the new thing must fit into almost 
any existing infrastructures like HamNet, Internet, Packet Radio, DXCluster, 
QSL Databases, DStar, DMR, APRS, and even Manual Operations, and systems I 
forgot about. For example, if someone types in a telegram in crzcq someone 
the receiver who may have his callbook entry on QRZ.com, can read it there. 
I have choosen these two projects as an example because they have their 
differences, and it illustrates perfectly that you to not go with a simple 
database. \\

If you do someone opens the next database because he does not feel 
comfortable with your work. However this may happen anyway, qtc-net can deal 
with it. \\

All this has lead me to the design of QTC-Net.  

\section{Architecture}

The main component of qtc net is a public key signed message, which is
distributed between all nodes\footnote{It except for the bockchain data, 
it is possible to filter down the amount of messages until the amount 
of traffic is low enough for your line}.\\

A message is one of the following types: 

\begin{description}
	\item[telegram] a standard text message
	\item[qsp] states if a telegram was received
	\item[operator] holds operator Information
	\item[key] holds an operators public key
	\item[revoke] points out if an operators key is revoked
	\item[trust] publishes a trustlevel for a public key
\end{description}


Every Message is signed with a public key owned by the creators callsign. 
Every Message is public, too. So everyone can read it. Because of the keys 
and trustlevels no extra authentication is needed if you are a node. \\

\subsection{Roles and Aliases}

A telegram contains 3 callsigns, the sender of the telegram (from), he 
receiver of the telegram (to) and the creator of the message (call). This 
makes it possible that the creator of a message picks up a telegram from any 
sender to any receiver, so that sender and receiver do not need to have a 
public key stored in the network.\\

The Advantage of those 3 role concept is that even RTTY, PSK31 and CW links 
can be used together with QTC Net Telegrams, on APRS for example, a message 
is send by the sender to a receiver, but signed and placed into QTC Net by 
one of the APRS Gateways callsigns.\\ 

The QSP (deliver without fee) Message has two callsigns, the receiver of 
the telegram that is delivered (to), and again the messages creator (call). It
may happen that a telegram to oe1src is delivered to dd5tt which is the same OP
but who just owns two callsigns. The qsp-to keeps track of this. \\

The operator message contains a set-of-aliases for the creator of the aliases 
message (call). For example oe1src and dd5tt but also oe1src-7 and 9a/oe1src 
are aliases of oe1src. This also means that if there is a resource oe1src-7 
on aprs it can receive and tell the gateway to qsp (mark as delivered) a qtc-net 
telegram for oe1src. \\

An operator can also follow several different other cals with the set-of-followings
list in the operator message.  If you follow a call, you get every message to that call 
listed  too when you query for messages to your call, but not as new messages, those 
messages are getting sorted into a timeline and timeline-new message list\\

An Operator can have more than one different public keys, one for each 
node he is working with. He needs to sign both of them against each other. \\

\subsection{Syncronization}

The Server Nodes can syncronize each other by using different protocols, right
now there only is an http based sync protocol defined and implemented, but this
will change in the near future. The current http based protocol may be used to 
sync the nodes over HamNet or the Internet. 

\section{Development}


\section{Installation}

\end{document}
